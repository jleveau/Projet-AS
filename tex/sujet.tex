\documentclass{report}
\usepackage{mathtools}
\usepackage{color}
\usepackage[utf8]{inputenc} 
\title{Projet d’Analyse Syntaxique
Université de Bordeaux, année 2014–2015
L3 Informatique et Math-Info.}
\begin{document}
\maketitle
\tableofcontents
\section{Modalités}{
Projet d’Analyse Syntaxique
Université de Bordeaux, année 2014–2015
L3 Informatique et Math-Info.
1 Modalités
Ce projet est à réaliser par groupes de 3 ou 4 étudiant(e)s. Chaque membre doit contribuer
de façon significative au
travail de programmation
, avoir connaissance des options choisies, des
difficultés rencontrées, et des solutions retenues. Il/Elle exposera sa contribution individuellement
lors d’une soutenance de 25mn (soutenances prévues la semaine du 11 mai).
Contrairement à d’autres projets (comme celui d’informatique théorique 2 ce semestre), on ne
fournit pas de code de départ. Le projet sera commencé en TD. Il laisse d’autre part beaucoup
d’options et de liberté à chaque groupe. Une partie est imposée, mais vous pourrez ajouter vos
propres extensions si le temps vous le permet.
La notation tiendra compte du degré de réalisation du sujet, de la qualité du code (portabilité,
lisibilité, documentation), du jeu de tests présenté, et de la qualité du rapport et de la soutenance.
Les notes peuvent varier à l’intérieur d’un groupe en cas de travail trop inégal.
En plus du code source (avec jeux de tests), on demande un rapport court présentant le travail
réalisé, illustré par des exemples, exposant les problèmes rencontrés et les choix effectués. Idéalement,
le rapport sera présenté au format
html
, et généré partiellement par votre logiciel. Le projet doit
être transmis à
aspp3@labri.fr
, avec comme sujet
[Projet ASPP3] Nom1 Nom2 Nom3 Nom4
.
Le mail devra contenir une archive au format .tar.gz. Cette archive une fois décompressée contiendra
un répertoire nommé Nom1-Nom2-Nom3-Nom4, qui lui-même contiendra les sources et le rapport.
Enfin, vous devez utiliser un système de gestion de projets, comme subversion (savanne au CREMI)
ou git.
Date
stricte
de remise du projet, code et rapport : 10 mai 2015, 12h00
}
\section{Objectif et Description}
{
Ce projet consiste à réaliser un logiciel permettant :
— de vérifier la syntaxe et certains points sémantique de programmes,
— de produire une documentation de ces programmes et de présenter les programmes et la documentation comme un document html (à la manière de Doxygen).
On ne demande bien sûr pas un logiciel aussi abouti que
Doxygen
. Par ailleurs, le projet permettra
d’avoir un document principal LATEX, ce qui n’est pas le fonctionnement de
Doxygen.
Ultimement, le programme développé devrait permettre de présenter le rapport du projet lui-
même. Le format des documents produits sera du HTML5 et il sera fait appel à plusieurs technologies
entourant ce type de document, comme CSS, Javascript (et des bibliothèques telles que jQuery),
MathML etc... Le  rojet doit être écrit en langage C, en utilisant flex et bison.



Le logiciel que vous écrirez prend en entrée un programme à documenter. Ce programme à
documenter est composé de plusieurs fichiers :
—
un fichier L
A
T
E
X optionnel, permettant de décrire et d’organiser la documentation.
—
des sources écrits dans le langage à analyser. Ces sources pourront contenir des commentaires
pour guider le système de documentation (à nouveau, comme le permet le logiciel
Doxygen
).
Dans ce projet, les sources à analyser seront eux-mêmes écrits dans les langages de développement :
C
, et si le temps vous le permet,
flex
,
bison
et
JavaScript
. De cette façon, il sera possible d’écrire
le rapport en utilisant le logiciel développé dans le projet.
Le projet se déroulera en plusieurs étapes.
—
Il s’agira dans un premier temps d’utiliser une grammaire du C pour transformer le code
C en document HTML. On ne demande pas d’écrire la grammaire. On peut utiliser pour
bison la grammaire C11 de
http://www.quut.com/c/ANSI-C-grammar-y.html
par exemple.
Un source flex est aussi disponible depuis la même page.
Le code devra être bien présenté, c’est-à-dire, proprement indenté et coloré. La présentation du
code doit permettre de facilement le parcourir et l’appréhender. Cela passe par l’implémentation
d’un certaines fonctionnalités, comme faire apparaître en surbrillance toutes les occurrences
d’une variable lorsque le pointeur de la souris passe sur son nom et pouvoir rejoindre l’endroit
du code où elle est déclarée lorsque l’on clique dessus.
Ces fonctionnalités sont décrites en Section


}

\end{document}
