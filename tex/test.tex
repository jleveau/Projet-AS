\documentclass{report}
\usepackage{mathtools}
\usepackage{color}
\title{Projet d’Analyse Syntaxique
Université de Bordeaux, année 2014–2015
L3 Informatique et Math-Info.}
\begin{document}
\maketitle
\tableofcontents
\part{Introduction}

\part{Projet}
\chapter{Présentation du Projet}
\chapter{Objectifs Atteints}
\chapter{Présentation du code C}
\chapter{Présentation de la documentation}
\chapter{Avis Personnel}
\part{Conclusion}
\section{section A}{eghzipeghez\underline{\color{red}tes}
srzahporfhzafzhizafhoizhaoh o\subsection{gf r}{fyu g\subsubsection{tototototototo}{yhofie}
} \subsection{gf r}{fyu}}

\section{section B}{rehuzir\subsection{rezarezarezarezarez}{reresqdfsqfds\color{red}qfdsqfdqsareza\subsubsection{rezareaz}{\paragraph{rte}{rearezar}}}}
\section{section C}{inetrnrezja\subsection{rkezop}{erzeza}}
\section{section D}{\paragraph{totototo}{rhezarz}\subsection{toteaz}{rza}}
\begin{itemize}
	 \item toto

	        \item tata    
\end{itemize}

\begin{equation}
3x+4^{4^{x*3+4/2 > 3}}
{\sum_{i=1 }^{n} 2x+1} 
10^{3*foz^{pa}}
{\prod_{i=1 }^{n} 2x+1}
3x+4x*3+4/2 > 3
\end{equation}
















\end{document}
